\documentclass{sciposter} 
\usepackage[T1]{fontenc}
\usepackage{indentfirst}
\usepackage[brazil]{babel}
\usepackage[utf8]{inputenc}
\usepackage{epsfig,color,pifont}
\usepackage{amsmath}
\usepackage{amssymb}
\usepackage{multicol}
\usepackage{graphicx}
\usepackage{xcolor}
\usepackage{comment}
\hyphenation{li-te-ra-tu-ra}
\setmargins[4.9cm] 

\definecolor{blue}{HTML}{09013e}
\definecolor{BoxCol2}{rgb}{1,1,1}
\definecolor{BoxCol}{HTML}{09013e}
\definecolor{SectionCol}{rgb}{1,1,1}


\usepackage{tabularx}
\newcolumntype{E}[1]{>{\raggedright\let\newline\\\arraybackslash\hspace{0pt}}m{#1}} %Alinha na esquerda
\newcolumntype{C}[1]{>{\centering\let\newline\\\arraybackslash\hspace{0pt}}m{#1}}%Alinha no centro
\newcolumntype{D}[1]{>{\raggedleft\let\newline\\\arraybackslash\hspace{0pt}}m{#1}}%Alinha na direita


\begin{document}

% Linha superior
\begin{center}
\textcolor{blue}{\rule{74cm}{0.2mm}} \end{center}
\begin{minipage}{0.12\textwidth}
\includegraphics[scale=0.5]{Brasao_UFC_02} 
\end{minipage} %Brasão da UFC
\begin{minipage}{0.15\textwidth}
\includegraphics[scale=0.65]{logo_encontros.jpeg}
\end{minipage} % Logo do EU
\hspace{0.7cm}
\begin{minipage}{0.5\textwidth}
\begin{flushleft}
\large UNIVERSIDADE FEDERAL DO CEARÁ\\
{\large DEPARTAMENTO DE ESTATÍSTICA E MATEMÁTICA APLICADA}\\
{\large CURSO DE ESTATÍSTICA}\\
{\large ENCONTROS UNIVERSITÁRIOS 2022}\\
{\large IX ENCONTRO DE PROGRAMAS DE EDUCAÇÃO TUTORIAL DA UFC}
%IX Encontro de Programas de Educação Tutorial
\end{flushleft}
\end{minipage}%
\begin{minipage}{0.15\textwidth}
\includegraphics[scale=1]{petlog}
\end{minipage} % Logo do PET
\vspace{-0.55cm}
% Linha inferior
\begin{center}
\textcolor{blue}{\rule{74cm}{0.2mm}} \end{center}



\begin{center}
\LARGE \bf{JORNADA DE MINICURSOS}
\end{center}
\vspace{0.2cm}
\begin{center}
 \large \textbf{Autor:} Antônio Arthur Silva de Lima \\
\large \textbf{Coautores:} Ane Lara Nascimento Lima, João Karlos Silva de Medeiros, Luiz Guilherme Pessoa Aguiar e Eliane Nunes Martins \\
\large \textbf{Tutor:} Ronald Targino Nojosa \\
\end{center}
\vspace{-0.2cm}
\begin{multicols}{2}
{\large
%%%%%%%%%%%%%%%%%%%%%%%%%%%%%%%%%%%%%%%%%%%%%%%%%%%%%%%%%%%%%%%%%%%%%%%%%%%%%%%%%%%%%%%%%%%
\vspace{0.35cm}
\section{Introdução}
Nesta era dos dados e de amplo uso das técnicas estatísticas nas mais diversas áreas, é de grande importância que o estatístico domine as ferramentas computacionais, mas também desenvolva as chamadas habilidades comportamentais (\textit{soft skills}), com o intuito de realizar tarefas de modo eficiente e de aprimorar a forma de se relacionar com o público geral. A Jornada de Minicursos promovida pelo PET Estatística busca desenvolver ou ampliar tais habilidades no grupo de petianos e levar conhecimentos das ferramentas computacionais, dentre outros, de forma compreensível para estudantes e público geral. 
Visando ampliar as ofertas da Jornada, o PET Estatística estabeleceu parcerias com outros grupos.\ Foram ofertados 9 minicursos, sendo 5 deles ministrados pelo PET Estatística e 4 pelos parceiros convidados: PET Computação, PET Engenharia Civil, PET Enfermagem e Sociedade de Debates da UFC. Além das parcerias, promovemos uma campanha para a doação de alimentos.\ Todos os minicursos foram abertos e gratuitos, sendo apenas a emissão de certificados condicionada a doação de alimentos: 1Kg para um minicurso, 2Kg para dois minicursos  e 3Kg para três ou mais minicursos.

\section{Objetivos}
 \subsection{Objetivo principal}
 A Jornada de Minicursos objetiva aprimorar e disseminar conhecimentos sobre temas de interesse na área de Estatística e que possam contribuir para a formação de todos os participantes.

 \subsection{Objetivos específicos}
 \begin{itemize}
 \item Estender o aprendizado das ferramentas computacionais para análise de dados, principalmente entre alunos da Estatística;
 \item Apresentar usos e práticas reais de recursos da Estatística;
 \item Promover a interdisciplinaridade da Estatística com outros programas e grupos da UFC;
 \item Despertar o interesse e a curiosidade dos estudantes pelos usos da Estatística;
 \item Desenvolver entre os bolsistas habilidades comportamentais.  
 \end{itemize}
\vspace{0.35cm}
 
\vspace{0.2cm}
\section{Metodologia}
%A Jornada de 2022 ofertou 9 minicursos, sendo 5 deles ministrados pelo PET Estatística e 4 por convidados: PET Computação, PET Engenharia Civil, PET Enfermagem e Sociedade de Debates da UFC. As inscrições para os mesmos deu-se através de formulários eletrônicos, disponibilizados através do Instagram do PET Estatística. As aulas foram todas ministradas à tarde em laboratórios e/ou em auditório, seguindo o planejamento realizado previamente pelos ministrantes. Além disso, atendendo à pedidos de vários alunos e visando maior alcance de inscritos, optamos por ministrar o minicurso de Power BI remotamente. Nesta edição, promovemos uma campanha para a doação de alimentos. Os minicursos foram abertos e gratuitos ao público geral, sendo apenas a emissão de certificados condicionada a doação de alimentos: 1Kg para um minicurso, 2Kg para dois minicursos  e 3Kg para três ou mais minicursos. Após o encerramento da Jornada, 153 Kg alimentos foram doados ao Lar Torres de Melo, instituição de acolhimento de idosos localizada em Fortaleza. 

Os petianos foram divididos em grupos e cada grupo se responsabilizou pelo estudo, preparação e apresentação de um minicurso.\ A Jornada foi divulgada no Instagram do PET Estatística e, para as inscrições, foi disponibilizado formulário através dos canais digitais do grupo.\ As aulas foram todas ministradas à tarde em laboratórios e/ou em auditório, seguindo o planejamento realizado previamente pelos ministrantes.\ Além disso, atendendo a pedido de vários alunos e visando maior alcance de inscritos, optamos por ministrar o minicurso de Power BI remotamente.\ No Quadro 1 é apresentado o cronograma da Jornada. 

\begin{center}
    Quadro 1: Cronograma da Jornada.

    \begin{table}[H]
        \centering
        \begin{tabular}{|l|c|}
        \hline
         \textbf{Minicurso} &  \textbf{Datas} \\ \hline
         Análise Exploratória de Dados c/ R  &  13/09, 15/09 e 16/09 \\
         \hline
         Git/Github  &  14/09 \\
         \hline
         Pirmeiros Socorros  &  20/09 \\
         \hline
         Investimentos  &  21/09 \\
         \hline
         Oratória  &  22/09 \\
         \hline
         Power BI  &  26/09, 27/09 e 28/09 \\
         \hline
         Análise Exploratória de Dados c/ Python  &  29/09, 30/09 e 03/10 \\
         \hline
         R/RStudio  &  04/10 e 07/10 \\
         \hline
         LaTex  &  06/10 \\
         \hline
        \end{tabular}
        \label{cronograma}
    \end{table}
\end{center}

\vspace{0.3cm}
\section{Aplicação e Resultados}
A Jornada de Minicursos de 2022 foi realizada entre os dias 13 de setembro e 07 de outubro.\ Os minicursos ministrados pelos convidados foram: Git/Github (PET Computação), Investimentos (PET Engenharia Civil), Oratória (Sociedade de Debates) e Primeiros Socorros (PET Enfermagem).\ O total de inscritos e a carga horária para cada minicurso são apresentados no Quadro 2. 

\begin{center}
    Quadro 2: Número de inscrições e carga horária dos minicursos.\vspace{0.5cm}
    \begin{table}[H]
    \centering
        \begin{tabular}{|l|c|c|}
        \hline
        \textbf{Minicurso} & \textbf{Inscritos} & \textbf{Tempo} \\ \hline
        Análise Exploratória de Dados c/ R & 25 & 6h \\
        \hline
        Git/Github & 29 & 2h \\ 
        \hline
        Primeiros Socorros & 28 & 2h \\
        \hline
        Investimentos & 29 & 2h \\
        \hline
        Oratória & 29 & 2h \\
        \hline
        Power BI & 48 & 6h \\
        \hline
        Análise Exploratória de Dados c/ Python & 35 & 6h \\
        \hline
        R/RStudio & 18 & 4h \\
        \hline
        LaTeX & 16 & 3h \\
        \hline
        \textbf{Total} & \textbf{257} & \textbf{33h} \\
        \hline
       \end{tabular}
       \label{frequencia}
    \end{table}
\end{center}

Ao todo, foram emitidos 157 certificados e arrecadados 153Kg de alimentos. Após a Jornada, todos os alimentos foram doados para o Lar Torres de Melo, em Fortaleza (Figura \ref{fig:Figura1}).

\begin{figure}[H] \normalsize
\begin{center}
\includegraphics[scale=0.40]{LarTorresDeMelo.jpeg}
\caption{Entrega de alimentos no Lar Torres de Melo}
\label{fig:Figura1}
\end{center}
\end{figure}
\vspace{-1cm}

Durante as aulas, tanto de minicursos do PET Estatística quanto dos convidados, houve intensa participação dos alunos, os quais sempre interagiam entre si e com os petianos a respeito do conteúdo. Na Figura \ref{fig:Figura2}, são apresentados registros de como e onde ocorreram as aulas em alguns dias ao longo da Jornada.

\begin{figure}[H]
    \centering
    \includegraphics[scale=0.3]{R.jpeg}
    \includegraphics[scale=0.3]{primeiros_socorros.jpeg}
    \caption{Análise Exploratória de Dados c/ R e Investimentos}
    \label{fig:Figura2}
\end{figure}
\vspace{-1cm}

Também é importante salientar que, ao final das aulas, foram disponibilizados materiais (PDF's e códigos dos programas computacionais) aos alunos que participaram dos minicursos do PET Estatística, a fim de que as atividades feitas em sala pudessem ser reproduzidas posteriormente pelos estudantes.\ Ainda, ao fim de cada minicurso, um formulário de \textit{feedback} foi enviado aos participantes para mensurar o impacto da Jornada.\ Com isso, observamos que dentre as 91 respostas da enquete, 100\% consideraram como excelente ou ótima a organização e o planejamento dos ministrantes, 96\% consideraram a abordagem e metodologia das apresentações como excelente ou ótima, e 93\% disseram que os minicursos contribuíram muito para o seu aprendizado.

\section{Conclusões}
\vspace{0.35cm}
A Jornada de Minicursos de 2022 contribuiu para o aprimoramento das competências exigidas de um estatístico, na medida em que trabalhou a interdisciplinaridade do profissional em vários âmbitos. Além disso, proporcionou ao grupo PET Estatística mais uma oportunidade de pôr em prática habilidades adquiridas com a formação, alcançando também vários discentes fora do grupo.

\vspace{0.20cm}

\end{multicols}
\textcolor{blue}{\rule{74cm}{0.2mm}}
\end{document}

